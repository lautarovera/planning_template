\documentclass[
11pt, % The default document font size, options: 10pt, 11pt, 12pt
%codirector, % Uncomment to add a codirector to the title page
]{charter} 




% El títulos de la memoria, se usa en la carátula y se puede usar el cualquier lugar del documento con el comando \ttitle
\titulo{Bootloader Multi-Etapa} 

% Nombre del posgrado, se usa en la carátula y se puede usar el cualquier lugar del documento con el comando \degreename
\posgrado{Carrera de Especialización en Sistemas Embebidos} 
%\posgrado{Carrera de Especialización en Internet de las Cosas} 
%\posgrado{Carrera de Especialización en Intelegencia Artificial}
%\posgrado{Maestría en Sistemas Embebidos} 
%\posgrado{Maestría en Internet de las cosas}

% Tu nombre, se puede usar el cualquier lugar del documento con el comando \authorname
\autor{Ing. Lautaro Juan Bautista Vera} 

% El nombre del director y co-director, se puede usar el cualquier lugar del documento con el comando \supname y \cosupname y \pertesupname y \pertecosupname
\director{Mg. Ing. Christian Yánez Flores}
\pertenenciaDirector{INTI, FIUBA} 
% FIXME:NO IMPLEMENTADO EL CODIRECTOR ni su pertenencia
%\codirector{John Doe} % para que aparezca en la portada se debe descomentar la opción codirector en el documentclass
%\pertenenciaCoDirector{FIUBA}

% Nombre del cliente, quien va a aprobar los resultados del proyecto, se puede usar con el comando \clientename y \empclientename
\cliente{Juan Pablo Trípodi}
\empresaCliente{ECCOSUR}

% Nombre y pertenencia de los jurados, se pueden usar el cualquier lugar del documento con el comando \jurunoname, \jurdosname y \jurtresname y \perteunoname, \pertedosname y \pertetresname.
\juradoUno{Nombre y Apellido (1)}
\pertenenciaJurUno{pertenencia (1)} 
\juradoDos{Nombre y Apellido (2)}
\pertenenciaJurDos{pertenencia (2)}
\juradoTres{Nombre y Apellido (3)}
\pertenenciaJurTres{pertenencia (3)}
 
\fechaINICIO{30 de abril de 2021}		%Fecha de inicio de la cursada de GdP \fechaInicioName
\fechaFINALPlan{18 de junio de 2021} 	%Fecha de final de cursada de GdP
\fechaFINALTrabajo{15 de mayo de 2022}	%Fecha de defensa pública del trabajo final


\begin{document}

\maketitle
\thispagestyle{empty}
\pagebreak


\thispagestyle{empty}
{\setlength{\parskip}{0pt}
\tableofcontents{}
}
\pagebreak


\section*{Registros de cambios}
\label{sec:registro}


\begin{table}[ht]
\label{tab:registro}
\centering
\begin{tabularx}{\linewidth}{@{}|c|X|c|@{}}
\hline
\rowcolor[HTML]{C0C0C0} 
Revisión & \multicolumn{1}{c|}{\cellcolor[HTML]{C0C0C0}Detalles de los cambios realizados} & Fecha      \\ \hline
0      & Creación del documento                                 				& 22/10/2021 \\ \hline
1      & Se completa hasta el punto 5 inclusive                 				& 04/11/2021 \\ \hline
2      & Correcciones de la versión 1. Se completa hasta el punto 9 inclusive	& 08/11/2021 \\ \hline
%		  Se puede agregar algo más \newline
%		  En distintas líneas \newline
%		  Así                                                    & dd/mm/aaaa \\ \hline
%3      & Se completa hasta el punto 11 inclusive                & dd/mm/aaaa \\ \hline
%4      & Se completa el plan	                                 & dd/mm/aaaa \\ \hline
\end{tabularx}
\end{table}

\pagebreak



\section*{Acta de constitución del proyecto}
\label{sec:acta}

\begin{flushright}
Buenos Aires, \fechaInicioName
\end{flushright}

\vspace{2cm}

Por medio de la presente se acuerda con el \authorname\hspace{1px} que su Trabajo Final de la \degreename\hspace{1px} se titulará ``\ttitle'', consistirá en el desarrollo de un bootloader multi-etapa con \textit{secureboot} y tendrá un presupuesto preliminar estimado de 600 hs de trabajo y \$200, con fecha de inicio \fechaInicioName\hspace{1px} y fecha de presentación pública \fechaFinalName.

Se adjunta a esta acta la planificación inicial.

\vfill

% Esta parte se construye sola con la información que hayan cargado en el preámbulo del documento y no debe modificarla
\begin{table}[ht]
\centering
\begin{tabular}{ccc}
\begin{tabular}[c]{@{}c@{}}Ariel Lutenberg \\ Director posgrado FIUBA\end{tabular} & \hspace{2cm} & \begin{tabular}[c]{@{}c@{}}\clientename \\ \empclientename \end{tabular} \vspace{2.5cm} \\ 
\multicolumn{3}{c}{\begin{tabular}[c]{@{}c@{}} \supname \\ Director del Trabajo Final\end{tabular}} \vspace{2.5cm} \\
%\begin{tabular}[c]{@{}c@{}}\jurunoname \\ Jurado del Trabajo Final\end{tabular}     &  & \begin{tabular}[c]{@{}c@{}}\jurdosname\\ Jurado del Trabajo Final\end{tabular}  \vspace{2.5cm}  \\
%\multicolumn{3}{c}{\begin{tabular}[c]{@{}c@{}} \jurtresname\\ Jurado del Trabajo Final\end{tabular}} \vspace{.5cm}                                                                     
\end{tabular}
\end{table}




\section{1. Descripción técnica-conceptual del proyecto a realizar}
\label{sec:descripcion}

Un bootloader es un programa cuyo principal propósito es permitir a los sistemas embebidos actualizar su software sin el uso de hardware especializado como puede ser un programador JTAG. En ciertos casos, puede ser utilizado como el punto más temprano para la validación de la integridad de un sistema embebido. Existen distintos tipos de bootloaders y pueden comunicarse a través de casi cualquier protocolo como USART, CAN, I2C, Ethernet, USB, entre otros.

El uso de bootloaders es mandatorio en la industria. Cumple una rol vital en el ciclo de vida del producto, precisamente en la etapa de mantenimiento, ya que permite actualizar el software para introducir parches y nuevas funcionalidades.

Los bootloaders en general son reutilizables y sólo dependen de la arquitectura del sistema y no de la aplicación o caso de uso.

El cliente ECCOSUR, auspiciante de este proyecto, requiere que el desarrollo tenga lugar dentro del ciclo de vida de su producto, un electrocardiógrafo ambulatorio Holter. Se tiene entonces en este proyecto que el caso de uso es el Holter de ECCOSUR

Respecto al Holter, su situación actual es la de un bootloader genérico USB, que cumple con la actualización del software mediante este protocolo. El firmware del Holter es bastante complejo, y el bootloader actual no está a la altura de las necesidades del producto. Se hace necesario una evolución hacia un bootloader de mayores prestaciones.

El presente proyecto propone una innovación del bootloader actual del Holter, desarrollando un bootloader que comprenda las siguientes funcionalidades:

\begin{itemize}
	\item Multi-etapa, con etapa primaria (\textit{bootmanager}) y etapa secundaria (\textit{bootloader} propiamente dicho).
	\item Capacidad de seleccionar el modo de operación mediante \textit{branching}, aplicación o \textit{bootloader}.
	\item Comunicación con un "Flasher" (host) a través del protocolo USB.
	\item Capacidad de actualización y restauración del firmware.
	\item Seguridad e integridad:
	\begin{itemize}
		\item Localización del bootloader en flash protegida.
		\item \textit{Secureboot}: sólo se ejecuta la aplicación si ésta es válida.
		\item \textit{Checksum}: verifica que no hubo cambios accidentales por pérdida de paquetes durante la transmisión de los paquetes.
	\end{itemize}
	\item Reutilizable, mediante un manual de integración un ingeniero idóneo debería ser capaz de incluir este bootloader en su proyecto.
\end{itemize}

En la figura 1 se puede observar el diagrama de bloques de un bootloader genérico de dos etapas. La primera etapa referida como ``Branch'' es el denominado \textit{bootmanager} y es donde se lleva a cabo el \textit{branching} y el \textit{secureboot}. La segunda etapa hace referencia al \textit{bootloader} y es donde se lleva a cabo la descarga de la imagen de software a través del protocolo correspondiente y su posterior actualización.

%\vspace{25px}

\begin{figure}[htpb]
\centering 
\includegraphics[width=.5\textwidth]{./Figuras/bootloader.png}
\caption{Diagrama en bloques del sistema.}
\label{fig:diagBloques}
\end{figure}

\vspace{25px}


\section{2. Identificación y análisis de los interesados}
\label{sec:interesados}

A continuación se identifican los roles de los interesados:

\begin{table}[ht]
%\caption{Identificación de los interesados}
%\label{tab:interesados}
\begin{tabularx}{\linewidth}{@{}|l|X|X|l|@{}}
\hline
\rowcolor[HTML]{C0C0C0} 
Rol           & Nombre y Apellido & Organización 	& Puesto	\\ \hline
Auspiciante   & \clientename      &\empclientename  & Gerente I+D+i		\\ \hline
Cliente       & \clientename      &\empclientename	& Gerente I+D+i     \\ \hline
%Impulsor      &                   &              	&        	\\ \hline
Responsable   & \authorname       & FIUBA        	& Alumno 	\\ \hline
%Colaboradores & Mg. Ing. Patricio Bos & FIUBA       & Profesor de Gestión de Proyectos	\\ \hline
Orientador    & \supname	      & \pertesupname 	& Director Trabajo final \\ \hline
%Equipo        & miembro1 \newline 
%				miembro2          &              	&        	\\ \hline
%Opositores    &                   &              	&        	\\ \hline
%Usuario final &                   &              	&        	\\ \hline
\end{tabularx}
\end{table}

Por otro lado, el análisis de las características de los interesados es el siguiente:

\begin{itemize}
	\item Auspiciante: Es proactivo a la hora de brindar recursos técnicos.
	\item Cliente: Tiene especial interés en la innovación en seguridad e integridad del nuevo bootloader.
	\item Responsable: Tiene experiencia en bootloaders pero no en la arquitectura del microcontrolador del Holter (Texas Instruments). Trabaja jornada completa, puede saturarse de tareas. Para evitar que esto ocurra es fundamental una buena planificación.
	\item Orientador: Tiene experiencia en la definición de proyectos. Puede dar buen soporte a la hora de definir requerimientos y el desglose de tareas.
\end{itemize}



\section{3. Propósito del proyecto}
\label{sec:proposito}

El propósito de este proyecto es desarrollar un bootloader robusto, seguro, versátil y reutilizable con comunicación USB. El mismo debe cumplir con las funcionalidades descritas en la descripción técnica (sección 1) y se validará integrándose con el Holter de Eccosur.

\section{4. Alcance del proyecto}
\label{sec:alcance}

El presente proyecto incluye:

\begin{itemize}
	\item Desarrollo de \textit{bootmanager}
	\item Desarrollo de \textit{bootloader}
	\item Adaptación del mapa de memoria de la \textit{aplicación} de Holter para que sea compatible con el nuevo bootloader.
	\item Desarrollo de script de prueba que simule la aplicación \textit{flasher} para validar el bootloader.
	\item Manual de integración incluido en la memoria técnica del proyecto.
\end{itemize}

Por otro lado, el presente proyecto no incluye:

\begin{itemize}
	\item Desarrollo de \textit{flasher} como aplicación de sistema operativo de propósito general.
	\item Funcionalidad \textit{Firmware Update Over The Air} (FUOTA).
\end{itemize}

\section{5. Supuestos del proyecto}
\label{sec:supuestos}

Para el desarrollo exitoso del proyecto se tienen las siguientes hipótesis:

\begin{itemize}
	\item El responsable tiene experiencia en bootloaders pero no con arquitecturas de Texas Instruments.
	\item El cliente compartirá todo el código fuente de la aplicación del dispositivo Holter.
	\item El conocimiento en Makefile y Linker Scripts serán claves a la hora del desarrollo del proyecto.
	\item El \textit{stack} de comunicación USB podrá ser directamente reutilizado.
	\item La memoria \textit{flash} del microcontrolador es suficiente para almacenar una imagen estable para la restauración y la imagen nueva para la actualización.
\end{itemize}

\section{6. Requerimientos}
\label{sec:requerimientos}

\begin{enumerate}
	\item Requerimientos funcionales
		\begin{enumerate}
			\item El proyecto será desarrollado y mantenido mediante control de versiones \textit{GIT}.
			\item El proyecto será desarrollado en lenguaje C, compilado con \textit{TI ARM-CGT 20.2.2.LTS} siguiendo la guía de estilos y analizado 
			estáticamente con la herramienta \textit{Splint}.
			\item El \textit{bootmanager} implementará \textit{branching} de manera segura.
			\item El \textit{bootmanager} implementará \textit{secureboot} para chequeos de origen de firmware antes de un salto a aplicación. 
			\item El \textit{bootloader} actualizará correctamente la versión de firmware actual por una nueva versión certificada 	
			(\textit{Update}).
			\item El \textit{bootloader} restaurará el sistema a la última versión estable instalada en caso de cualquier defecto en la 
			instalación o en el funcionamiento de nuevas versiones \textit{Rollback}.
			\item El \textit{bootloader} soportará comunicación USB para descargar imágenes de firmware.
			\item El \textit{bootloader} soportará el formato de imagen binaria \textit{Motorola SREC}.
			\item El \textit{bootloader} implementará funciones de redundancia cíclica, \textit{CRC} y \textit{checksum} tanto durante la descarga 
			USB como la instalación.
		\end{enumerate}
	\item Requerimientos de documentación
		\begin{enumerate}
			\item El código será documentado con la herramienta \textit{Doxygen}.
			\item El ciclo de vida del presente proyecto será respaldado por una memoria técnica.
			\item La integración del bootloader en otras aplicaciones y casos de uso será facilitada mediante un manual de integración.
		\end{enumerate}
	\item Requerimiento de testing
		\begin{enumerate}
			\item Se desarrollará un script en lenguaje \textit{Python} que cumpla con la funcionalidad mínima de \textit{host} para la descarga USB 
			de la imagen.
			\item Se validará el correcto funcionamiento del bootloader multi-etapa utilizando el script de \textit{Python} como \textit{host}.
		\end{enumerate}
	\item Requerimientos opcionales
		\begin{enumerate}
			\item En caso de que se desee un \textit{bootmanager} y \textit{bootloader}, serán dejados fuera de la sección protegida de la memoria.
		\end{enumerate}
\end{enumerate}

\section{7. Historias de usuarios (\textit{Product backlog})}
\label{sec:backlog}

\begin{table}[ht]
\begin{tabularx}{\linewidth}{@{}|X|c|c|@{}}
\hline
\rowcolor[HTML]{C0C0C0}
Historia de usuario 															& Prioridad & Ponderación	\\ \hline
Como cliente, quiero un bootloader que actualice imágenes de forma segura. 	  	& 8        	& 10      		\\ \hline
Como cliente, quiero que el bootloader se valide con un aplicación "host".    	& 9        	& 7       		\\ \hline
Como responsable, quiero asegurar la integridad de los datos y la instalación. 	& 10       	& 10      		\\ \hline
Como técnico integrador, quiero poder reutilizar el bootloader.               	& 5        	& 6       		\\ \hline
\end{tabularx}
\end{table}

\section{8. Entregables principales del proyecto}
\label{sec:entregables}

\begin{itemize}
	\item Código fuente del bootloader multi-etapa.
	\item Código fuente del script \textit{host}.
	\item Documentación HTML del código con \textit{Doxygen}. 
	\item Manual de integración.
	\item Memoria Técnica.
\end{itemize}

\section{9. Desglose del trabajo en tareas}
\label{sec:wbs}

\begin{enumerate}
\item Planificación
	\begin{enumerate}
	\item Toma de requerimientos (10 hs).
	\item Análisis de requerimientos (40 hs).
	\item Desarrollo de documentación relativa a la planificación (10 hs).
	\end{enumerate}
\item Estudio
	\begin{enumerate}
	\item Arquitectura del MCU TI MSP430FR5994 (40 hs).
	\item Makefile (6 hs).
	\item CMake (6 hs).
	\item Linker scripts (6 hs).
	\end{enumerate}
\item Diseño del \textit{bootmanager}
	\begin{enumerate}
	\item \textit{Branching} (2 hs).
	\item \textit{Secureboot} (40 hs).
	\item Máquina de estados finitos (40 hs).
	\item Componente NIR (20 hs).
	\end{enumerate}
\item Diseño del \textit{bootloader}
	\begin{enumerate}
	\item \textit{Update} (40 hs).
	\item \textit{Rollback} (40 hs).
	\item \textit{CRC} y \textit{Checksum} (40 hs).
	\item Integración del stack USB del \textit{legacy} bootloader (10 hs).
	\item Máquina de estados finitos (40 hs).
	\end{enumerate}
\item Codificación y depuración
	\begin{enumerate}
	\item Makefile/CMake scripts (20 hs).
	\item Linker scripts (20 hs).
	\item \textit{Bootmanager} (10 hs).
	\item \textit{Bootloader} (30 hs).
	\item Comentarios \textit{Doxygen} y páginas \textit{Markdown} (6 hs).
	\end{enumerate}
\item Script \textit{host}
	\begin{enumerate}
	\item Estudio (10 hs).
	\item Diseño (10 hs).
	\item Implementación (10 hs).
	\end{enumerate}
\item Validación
	\begin{enumerate}
	\item Test de \textit{Upload} (6 hs).
	\item Test de \textit{Update} (6 hs).
	\item Test de \textit{Rollback} (10 hs).
	\item Test de seguridad (40 hs).
	\end{enumerate}
\item Documentación
	\begin{enumerate}
	\item Documentación HTML con \textit{Doxygen} (6 hs).
	\item Manual de integración en \textit{Markdown} (6 hs).
	\item Memoria técnica (40 hs).
	\end{enumerate}
\end{enumerate}


Cantidad total de horas: (620 hs)

\section{10. Diagrama de Activity On Node}
\label{sec:AoN}

\begin{consigna}{red}
Armar el AoN a partir del WBS definido en la etapa anterior. 

%La figura \ref{fig:AoN} fue elaborada con el paquete latex tikz y pueden consultar la siguiente referencia \textit{online}:

%\url{https://www.overleaf.com/learn/latex/LaTeX_Graphics_using_TikZ:_A_Tutorial_for_Beginners_(Part_3)\%E2\%80\%94Creating_Flowcharts}

\end{consigna}

\begin{figure}[htpb]
\centering 
\includegraphics[width=.8\textwidth]{./Figuras/AoN.png}
\caption{Diagrama en \textit{Activity on Node}}
\label{fig:AoN}
\end{figure}

Indicar claramente en qué unidades están expresados los tiempos.
De ser necesario indicar los caminos semicríticos y analizar sus tiempos mediante un cuadro.
Es recomendable usar colores y un cuadro indicativo describiendo qué representa cada color, como se muestra en el siguiente ejemplo:



\section{11. Diagrama de Gantt}
\label{sec:gantt}

\begin{consigna}{red}

Existen muchos programas y recursos \textit{online} para hacer diagramas de gantt, entre los cuales destacamos:

\begin{itemize}
\item Planner
\item GanttProject
\item Trello + \textit{plugins}. En el siguiente link hay un tutorial oficial: \\ \url{https://blog.trello.com/es/diagrama-de-gantt-de-un-proyecto}
\item Creately, herramienta online colaborativa. \\\url{https://creately.com/diagram/example/ieb3p3ml/LaTeX}
\item Se puede hacer en latex con el paquete \textit{pgfgantt}\\ \url{http://ctan.dcc.uchile.cl/graphics/pgf/contrib/pgfgantt/pgfgantt.pdf}
\end{itemize}

Pegar acá una captura de pantalla del diagrama de Gantt, cuidando que la letra sea suficientemente grande como para ser legible. 
Si el diagrama queda demasiado ancho, se puede pegar primero la ``tabla'' del Gantt y luego pegar la parte del diagrama de barras del diagrama de Gantt.

Configurar el software para que en la parte de la tabla muestre los códigos del EDT (WBS).\\
Configurar el software para que al lado de cada barra muestre el nombre de cada tarea.\\
Revisar que la fecha de finalización coincida con lo indicado en el Acta Constitutiva.

En la figura \ref{fig:gantt}, se muestra un ejemplo de diagrama de gantt realizado con el paquete de \textit{pgfgantt}. En la plantilla pueden ver el código que lo genera y usarlo de base para construir el propio.

\begin{figure}[htbp]
\begin{center}
\begin{ganttchart}{1}{12}
  \gantttitle{2020}{12} \\
  \gantttitlelist{1,...,12}{1} \\
  \ganttgroup{Group 1}{1}{7} \\
  \ganttbar{Task 1}{1}{2} \\
  \ganttlinkedbar{Task 2}{3}{7} \ganttnewline
  \ganttmilestone{Milestone o hito}{7} \ganttnewline
  \ganttbar{Final Task}{8}{12}
  \ganttlink{elem2}{elem3}
  \ganttlink{elem3}{elem4}
\end{ganttchart}
\end{center}
\caption{Diagrama de gantt de ejemplo}
\label{fig:gantt}
\end{figure}


\begin{landscape}
\begin{figure}[htpb]
\centering 
\includegraphics[height=.85\textheight]{./Figuras/Gantt-2.png}
\caption{Ejemplo de diagrama de Gantt rotado}
\label{fig:diagGantt}
\end{figure}

\end{landscape}

\end{consigna}


\section{12. Presupuesto detallado del proyecto}
\label{sec:presupuesto}

\begin{consigna}{red}
Si el proyecto es complejo entonces separarlo en partes:
\begin{itemize}
	\item Un total global, indicando el subtotal acumulado por cada una de las áreas.
	\item El desglose detallado del subtotal de cada una de las áreas.
\end{itemize}

IMPORTANTE: No olvidarse de considerar los COSTOS INDIRECTOS.

\end{consigna}

\begin{table}[htpb]
\centering
\begin{tabularx}{\linewidth}{@{}|X|c|r|r|@{}}
\hline
\rowcolor[HTML]{C0C0C0} 
\multicolumn{4}{|c|}{\cellcolor[HTML]{C0C0C0}COSTOS DIRECTOS} \\ \hline
\rowcolor[HTML]{C0C0C0} 
Descripción &
  \multicolumn{1}{c|}{\cellcolor[HTML]{C0C0C0}Cantidad} &
  \multicolumn{1}{c|}{\cellcolor[HTML]{C0C0C0}Valor unitario} &
  \multicolumn{1}{c|}{\cellcolor[HTML]{C0C0C0}Valor total} \\ \hline
 &
  \multicolumn{1}{c|}{} &
  \multicolumn{1}{c|}{} &
  \multicolumn{1}{c|}{} \\ \hline
 &
  \multicolumn{1}{c|}{} &
  \multicolumn{1}{c|}{} &
  \multicolumn{1}{c|}{} \\ \hline
\multicolumn{1}{|l|}{} &
   &
   &
   \\ \hline
\multicolumn{1}{|l|}{} &
   &
   &
   \\ \hline
\multicolumn{3}{|c|}{SUBTOTAL} &
  \multicolumn{1}{c|}{} \\ \hline
\rowcolor[HTML]{C0C0C0} 
\multicolumn{4}{|c|}{\cellcolor[HTML]{C0C0C0}COSTOS INDIRECTOS} \\ \hline
\rowcolor[HTML]{C0C0C0} 
Descripción &
  \multicolumn{1}{c|}{\cellcolor[HTML]{C0C0C0}Cantidad} &
  \multicolumn{1}{c|}{\cellcolor[HTML]{C0C0C0}Valor unitario} &
  \multicolumn{1}{c|}{\cellcolor[HTML]{C0C0C0}Valor total} \\ \hline
\multicolumn{1}{|l|}{} &
   &
   &
   \\ \hline
\multicolumn{1}{|l|}{} &
   &
   &
   \\ \hline
\multicolumn{1}{|l|}{} &
   &
   &
   \\ \hline
\multicolumn{3}{|c|}{SUBTOTAL} &
  \multicolumn{1}{c|}{} \\ \hline
\rowcolor[HTML]{C0C0C0}
\multicolumn{3}{|c|}{TOTAL} &
   \\ \hline
\end{tabularx}%
\end{table}


\section{13. Gestión de riesgos}
\label{sec:riesgos}

\begin{consigna}{red}
a) Identificación de los riesgos (al menos cinco) y estimación de sus consecuencias:
 
Riesgo 1: detallar el riesgo (riesgo es algo que si ocurre altera los planes previstos de forma negativa)
\begin{itemize}
	\item Severidad (S): mientras más severo, más alto es el número (usar números del 1 al 10).\\
	Justificar el motivo por el cual se asigna determinado número de severidad (S).
	\item Probabilidad de ocurrencia (O): mientras más probable, más alto es el número (usar del 1 al 10).\\
	Justificar el motivo por el cual se asigna determinado número de (O). 
\end{itemize}   

Riesgo 2:
\begin{itemize}
	\item Severidad (S): 
	\item Ocurrencia (O):
\end{itemize}

Riesgo 3:
\begin{itemize}
	\item Severidad (S): 
	\item Ocurrencia (O):
\end{itemize}


b) Tabla de gestión de riesgos:      (El RPN se calcula como RPN=SxO)

\begin{table}[htpb]
\centering
\begin{tabularx}{\linewidth}{@{}|X|c|c|c|c|c|c|@{}}
\hline
\rowcolor[HTML]{C0C0C0} 
Riesgo & S & O & RPN & S* & O* & RPN* \\ \hline
       &   &   &     &    &    &      \\ \hline
       &   &   &     &    &    &      \\ \hline
       &   &   &     &    &    &      \\ \hline
       &   &   &     &    &    &      \\ \hline
       &   &   &     &    &    &      \\ \hline
\end{tabularx}%
\end{table}

Criterio adoptado: 
Se tomarán medidas de mitigación en los riesgos cuyos números de RPN sean mayores a...

Nota: los valores marcados con (*) en la tabla corresponden luego de haber aplicado la mitigación.

c) Plan de mitigación de los riesgos que originalmente excedían el RPN máximo establecido:
 
Riesgo 1: plan de mitigación (si por el RPN fuera necesario elaborar un plan de mitigación).
  Nueva asignación de S y O, con su respectiva justificación:
  - Severidad (S): mientras más severo, más alto es el número (usar números del 1 al 10).
          Justificar el motivo por el cual se asigna determinado número de severidad (S).
  - Probabilidad de ocurrencia (O): mientras más probable, más alto es el número (usar del 1 al 10).
          Justificar el motivo por el cual se asigna determinado número de (O).

Riesgo 2: plan de mitigación (si por el RPN fuera necesario elaborar un plan de mitigación).
 
Riesgo 3: plan de mitigación (si por el RPN fuera necesario elaborar un plan de mitigación).

\end{consigna}


\section{14. Gestión de la calidad}
\label{sec:calidad}

\begin{consigna}{red}
Para cada uno de los requerimientos del proyecto indique:
\begin{itemize} 
\item Req \#1: copiar acá el requerimiento.

\begin{itemize}
	\item Verificación para confirmar si se cumplió con lo requerido antes de mostrar el sistema al cliente. Detallar 
	\item Validación con el cliente para confirmar que está de acuerdo en que se cumplió con lo requerido. Detallar  
\end{itemize}

\end{itemize}

Tener en cuenta que en este contexto se pueden mencionar simulaciones, cálculos, revisión de hojas de datos, consulta con expertos, mediciones, etc.  Las acciones de verificación suelen considerar al entregable como ``caja blanca'', es decir se conoce en profundidad su funcionamiento interno.  En cambio, las acciones de validación suelen considerar al entregable como ``caja negra'', es decir, que no se conocen los detalles de su funcionamiento interno.

\end{consigna}

\section{15. Procesos de cierre}    
\label{sec:cierre}

\begin{consigna}{red}
Establecer las pautas de trabajo para realizar una reunión final de evaluación del proyecto, tal que contemple las siguientes actividades:

\begin{itemize}
	\item Pautas de trabajo que se seguirán para analizar si se respetó el Plan de Proyecto original:
	 - Indicar quién se ocupará de hacer esto y cuál será el procedimiento a aplicar. 
	\item Identificación de las técnicas y procedimientos útiles e inútiles que se emplearon, y los problemas que surgieron y cómo se solucionaron:
	 - Indicar quién se ocupará de hacer esto y cuál será el procedimiento para dejar registro.
	\item Indicar quién organizará el acto de agradecimiento a todos los interesados, y en especial al equipo de trabajo y colaboradores:
	  - Indicar esto y quién financiará los gastos correspondientes.
\end{itemize}

\end{consigna}


\end{document}
